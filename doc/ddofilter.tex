\documentclass[11pt,a4paper]{article}
\usepackage[utf8]{inputenc}
\usepackage[T1]{fontenc}
\usepackage{amsmath}
\usepackage{amsfonts}
\usepackage{amssymb}
\usepackage{graphicx}
\usepackage[left=2.00cm, right=2.00cm, top=2.00cm, bottom=2.00cm]{geometry}
\author{GA}
\title{Driven Damped Oscillator (DDO) filter}
\begin{document}

\maketitle

A damped oscillator driven by a signal can be used as a filter and differentiator.
 
The relevant equation is
\begin{equation}\label{key}
m\ddot{x} = -k\,(x-u) - 2\gamma\, \dot{x}
\end{equation}	
where $u$ is the driving signal.

Inserting the frequency $\omega_0 = \sqrt{k/m}$ and scaling time to $\omega_0$, i.e. $t \rightarrow \omega_0 t$, we get
\begin{equation}\label{key}
\ddot{x} + 2 \zeta \dot{x}  + x = u 
\end{equation}	
where $\zeta = \gamma / m\omega_0$ is a dimensionless damping parameter (large $\zeta$ = high damping). The ``quality factor'' $Q$, sometimes used to characterize resonances, is the inverse of $\zeta$.

\section{ODE system}

The equation can be written as a 1st order ODE system 
\begin{equation}
\dot{\mathbf{x}} = \mathbf{A \cdot x} + \mathbf{b}\,u
\end{equation}

\begin{equation}
\mathbf{x} =
\left[\begin{matrix}
x \\
\dot{x}
\end{matrix}\right],
\quad
\mathbf{A} =
\left[\begin{matrix}
0 & 1 \\
-1 & -2\zeta
\end{matrix}\right],
\quad
\mathbf{b} =
\left[\begin{matrix}
0 \\
1
\end{matrix}\right]
\end{equation}

The solution is
\begin{equation}
\mathbf{x}(t) = e^{\mathbf{A}t}\mathbf{\cdot x}_0 + 
e^{\mathbf{A}t} \int_0^t dt' 
\left[e^{-\mathbf{A}t}\mathbf{\cdot b}\right] u
\end{equation}

The matrix exp can be obtained by finding the eigensystem of $\mathbf{A}$.

The eigenvalues of $\mathbf{A}$ are
\begin{equation}
\lambda_{1,2} = -\zeta \pm \sqrt{\zeta^2-1}
\end{equation}

For $1 > \zeta > 0$ we have complex (oscillatory) eigenvalues and the system is ``under-damped'' while for $ \zeta > 1$ the eigenvalues are real and the system is ``over-damped''.

The solution for constant $u$ can be written 
\begin{equation}
\mathbf{x}(t) = e^{\mathbf{A}t}\mathbf{\cdot x}_0 + 
u \, \mathbf{h}(t), 
\quad
\mathbf{h}(t) = \int_0^t dt'\; e^{\mathbf{A}t'}\mathbf{\cdot b}
\end{equation}

For $\zeta>1$ we have

\begin{equation}
q = \sqrt{\zeta^2-1}
\end{equation}

\begin{equation}
e^{\mathbf{A}t} = e^{-\zeta t} \cosh qt \cdot \mathbf{1} +
e^{-\zeta t}\frac{\sinh qt}{q} \cdot
\left[\begin{matrix}
\zeta & 1 \\
-1 & -\zeta
\end{matrix}\right]
\end{equation}

\begin{equation}
\mathbf{h(t)} = 
\left( 1 - e^{-\zeta t} \cosh qt \right)\cdot
\left[\begin{matrix}
1 \\
0
\end{matrix}\right] 
+ e^{-\zeta t}\frac{\sinh qt}{q}\cdot
\left[\begin{matrix}
-\zeta \\
1 
\end{matrix}\right]
\end{equation}

For $0 < \zeta <1$ we set $q=\sqrt{1-\zeta^2}$ and change $\cosh, \sinh$ to $\cos, \sin$.

For the special case $\zeta=1$ the equations simplify to
\begin{equation}
e^{\mathbf{A}t} = e^{-t}\cdot
\left[\begin{matrix}
1+t & t \\
-t & 1-t
\end{matrix}\right]
\end{equation}

\begin{equation}
\mathbf{h(t)} = 
\left[\begin{matrix}
 1 - e^{-t} - t\, e^{-t} \\
 t\,e^{-t}
\end{matrix}\right] 
\end{equation}

\section{Transfer function}

\begin{equation}\label{key}
G = \frac{x}{u}
\end{equation}

Applying standard Laplace transforms:
\begin{equation}\label{key}
G(s) = \frac{1}{1 + 2\zeta \, s + s^2}, \quad G(i\omega) = \frac{1}{1 + 2i \zeta \, \omega - \omega^2}
\end{equation}

\begin{equation}\label{key}
|G(i\omega)| = \frac{1}{\sqrt{(\omega^2-1)^2 + 4\zeta^2\omega^2}}
\end{equation}

\begin{equation}\label{key}
\arg G(i\omega) = \frac{2\zeta\omega}{\omega^2-1}
\end{equation}

It can be shown from the structure of $|G(i\omega)|$ that the value $\zeta = 1/\sqrt{2}$ is the lowest damping factor that does not make any signal amplification close to the resonance. The tranfer function is then $|G(i\omega)|=(\omega^4+1)^{-1/2}$, which is always lower than 1.

\section{Application to signal filtering}

Assuming we have a signal $u(t)$ digitized at a regular interval $\Delta t$, $u_k = u(k\cdot \Delta t)$ then we update the system by
\begin{equation}\label{key}
\mathbf{x}_{k+1} =   e^{\mathbf{A} \omega_0 \Delta t}\mathbf{\cdot x}_k + 
u_k \, \mathbf{h}(\omega_0 \Delta t)
\end{equation}

Then $[\mathbf{x}_{k+1}]_1$ corresponds to the filtered signal and  $\omega_0\, [\mathbf{x}_{k+1}]_2$ to the 1st order time derivative $du/dt$.

Li \& Ma (2014, DOI 10.1007/s00034-013-9634-z) "A New Approach for Filtering and Derivative Estimation of Noisy Signals"
use the above equation to filter a signal and extract the 1st order derivative. They do not state explicitly that it is the oscillator equation but arrive at the same results from a different reasoning. 

They use $\zeta = 1/\sqrt{2}$ and define a parameter $\epsilon = \sqrt{2}/\omega_0$. They write the equation
\begin{equation}\label{key}
\ddot{x} = -\frac{2}{\epsilon^2} (x-u) - \frac{2}{\epsilon}\dot{x}
\end{equation}





	
\end{document}
